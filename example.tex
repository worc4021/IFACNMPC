\section{Example}\label{sec:example}
This section we present a numerical simulation in which the state-dependent uncertainty enters through an upper bound on the linearisation
error for a nonlinear system. We consider an inverted pendulum described by $ml^2\ddot\varphi = mlg\sin\varphi+M$, where
$m=0.1$\,kg is the mass, $l=0.3$\,m is the length of the pendulum, $g=9.81$\,m\,s$\mbox{}^{-2}$ is the gravitational
constant and $M$ is the direct torque input. Discretisation using Euler backward finite differences 
at $f = 20$\,Hz yields
\begin{equation}
  \begin{split}
    x_1[k+1] &= x_1[k] + \frac{1}{f} x_2[k]\\
    x_2[k+1] &= \frac{g}{fl} \sin(x_1[k]) + x_2[k] + \frac{1}{fml^2}u[k]
  \end{split}
\end{equation}
where $x_1[k] = \varphi(t_0+k/f),x_2[k] = \dot\varphi(t_0+k/f)$ and $u[k] = M(t_0+k/f)$.
The nonlinearity is given by the sine function, and we can explicitly state the 
linearisation error as $e(x_1)=\frac{g}{fl}(\sin x_1 - x_1)$. Since the linearisation
only approximates the nonlinear dynamics around the equilibrium we can choose a closed interval 
$0\in D\subset\mathbb R$ and maximise the upper bound 
\begin{equation}
  \Delta e = \max_{x_1\in D}\abs{\frac{de}{dx_1}} = \frac{g}{fl}\max_{x_1\in D}\abs{\cos x_1-1}.
\end{equation}
This results in the linear system
\begin{equation}
  x^+ = \underbrace{\left(\begin{array}{cc}
  1 & f^{-1}\\ g(fl)^{-1} & 1
  \end{array}\right)}_A x + \underbrace{\left(\begin{array}{c} 0 \\(ml^2f)^{-1} \end{array}\right)}_B u
  +\left(\begin{array}{c}w_1\\ w_2\end{array}\right)
\end{equation}
with $\abs{w_1}\leq w_{1,\max}$ and $\abs{w_2}\leq\max\{w_{2,\max},\Delta e \abs{x_1}\}$.
We also introduce the input constraint $\abs{u}\leq3$\,N\,m and use the interval $D = [-\frac{3\pi}{4},
\frac{3\pi}{4}]$ and a horizon of $N=7$ to obtain the sequence of state constraints $\mathcal X_m$
depicted in Figure~\ref{fig:numerical:example}. The line search starting at the origin and
exploring towards $\mathpzc x_e=(-6,50)$ produces the optimal trajectories at the active-set switches
which are illustrated in blue. In pink a warm-started line search terminates on the boundary of the
feasible set before reaching the (infeasible) state $\mathpzc x_e=(0,30)$.